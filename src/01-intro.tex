\begin{center}
	\bfseries\fontsize{16}{19}\selectfont ВВЕДЕНИЕ
\end{center}
\addcontentsline{toc}{section}{ВВЕДЕНИЕ}

На данный момент на долю социальных сетей приходится треть всей рекламы в интернете. 

Социальная сеть - это платформа, которую формируют несколько человек. Люди делятся контентом, новостями, рекламой и другой информацией. Обмен всеми видами информации может осуществляться для информирования всех в различных целях. Эти цели могут быть разного типа, например, реклама и продажа товаров, социальные связи, хобби, работа и т.д. 

Для достижения этих целей может оказаться очень полезным формирование групп с одинаковыми желаниями. Каждый, у кого есть друзья и родственники в социальных сетях, а также знакомые подписчики, может создать группу.

Кроме того, социальные сети являются популярной платформой для рекламы. Владельцы групп могут использовать их для продвижения своих товаров или услуг. Это может быть как маленький бизнес, который хочет привлечь новых клиентов, так и крупная компания, рекламирующая свои продукты или проводящая акции.

Таргетированная реклама и кластеризация занимают прочные позиции в области продвижения различных товаров. Данная отрасль развивается быстрыми темпами и важно успеть отслеживать изменения в этой сфере.

Исследования в этой области привлекли значительное внимание по двум основным причинам. 

Во-первых, объем информации о товаре, доступной клиентам, постоянно растет, и поэтому желательно помочь клиентам разобраться в огромном количестве этой информации, чтобы найти наиболее подходящий им продукт или услугу. 

Во-вторых, понимание различных потребностей текущих и потенциальных клиентов является неотъемлемой частью управления взаимоотношениями с клиентами.

Возможность точного, а также эффективного определения потребности клиентов и, в результате, выдачи им рекламы товаров, которые они сочтут желательными, открывает огромные возможности для роста бизнеса.

Например, если компания проводит рекламную кампанию на продажу спортивной одежды, она может использовать кластеризацию, чтобы выделить группы пользователей, интересующихся спортом или фитнесом. Затем компания может настроить свои рекламные объявления таким образом, чтобы они были более релевантными для каждой группы, повышая эффективность рекламной кампании и увеличивая вероятность привлечения потенциальных клиентов.

В последнее время для точного изучения предпочтений пользователей и атрибутов товаров широко используются модели, основанные на машинном обучении. Преимущество машинного обучения в том, что оно может точно фиксировать представления о пользователях.

Применение нейронной сети в маркетинговой деятельности позволит выдавать наиболее подходящие товары, рекламные продукты, услуги непосредственному клиенту, что позволит повысить эффективность методов стимулирования сбыта и будет являться фактором устойчивого функционирования предприятия на рынке в условиях жесткой конкурентной борьбы, неопределенности и влияния значительных внешних факторов на его деятельность.

Цель данной работы -- разработка программного обеспечения для кластеризации пользователей социальных сетей с использованием нейронных сетей для обеспечения таргетированной рекламы.

Для достижения поставленной цели необходимо решить следующие задачи:
\begin{itemize}[leftmargin=1.6\parindent]
	\item[1)] провести анализ существующих методов персонализации пользователей социальных сетей;
	\item[2)] изучить способы получения информации о пользователях социальных сетей;
	\item[3)] разработать метод кластеризации пользователей социальных сетей  на основе модифицированной нейронной сети;
	\item[4)] разработать программное обеспечение, реализующее этот метод;
	\item[5)] провести исследование применимости разработанного программного обеспечения.
\end{itemize}

\pagebreak